\section{Source Code}


To save pages, only codes that are relevant to the text are given below. I will upload all the codes to \href{https://github.com/iChunyu/signal-process-demo}{my GitHub repository} after I get score of \textsl{Digital Signal Processing} to avoid being judged cheating.




\subsection{LPSD Algorithm}


{\noindent \bfseries Python Version}


\begin{lstlisting}[language=Python]
'''
Personal fuction library in Python
Contents
    (1) pxx,f = lpsd(data,fs,Jdes=1000)

Last Update: 2021-02-10
'''

# %%
import numpy as np
import matplotlib.pyplot as plt

# %% function01: LPSD
'''
LPSD: Logarithmic frequency axis Power Spectral Density
pxx,f = lpsd(data,fs,Jdes=1000)
   data --- Input data series, one dimension vector
   fs   --- Sample frequency, unit: Hz
   Jdes --- Desired frequency points, default: 1000
   pxx  --- One-sided PSD, unit: *^2/Hz
   f    --- Frequency points related to PSD points, unit: Hz
   Default window function is hanning window.

Ref: Improved spectrum estimation from digitized timeseries on a logarithmic frequency axis
     Article DOI: 10.1016/j.measurement.2005.10.010

Remark: Unfamiliar with Python by now, the functiondoesn't support matrix data by now.
    [MATLAB version supports matrix data.]

XiaoCY 2021-02-10
'''

# subfuction
# calculate frequency points
def getFreqs(N,fs,Jdes,Kdes,ksai):
    fmin = fs/N
    fmax = fs/2
    r_avg = fs/N*(1+(1-ksai)*(Kdes-1))

    g = (N/2)**(1/(Jdes-1))-1

    f = np.zeros(Jdes)-1
    L = np.copy(f)
    m = np.copy(f)
    j = 0
    fj = fmin
    while fj < fmax:
        rj = fj*g
        if rj < r_avg:
            rj = np.sqrt(rj*r_avg)
        if rj < fmin:
            rj = fmin
        
        Lj = np.floor(fs/rj)
        rj = fs/Lj
        mj = fj/rj

        f[j] = fj
        L[j] = Lj
        m[j] = mj

        fj = fj+rj
        j += 1
    
    idx = f<0
    f = np.delete(f,idx)
    L = np.delete(L,idx)
    m = np.delete(m,idx)
    return f,L,m

# main
def lpsd(data,fs,Jdes=1000):
    Kdes = 100
    ksai = 0.5
    N = len(data)

    f,L,m = getFreqs(N,fs,Jdes,Kdes,ksai)

    J = len(f)
    pxx = np.zeros(J)
    for j in range(J):
        Dj = np.floor((1-ksai)*L[j])
        Kj = np.floor((N-L[j])/Dj+1)
        w = np.hanning(L[j])
        C_PSD = 2/fs/np.dot(w,w)
        l = np.arange(0,L[j])
        W1 = np.cos(-2*np.pi*m[j]/L[j]*l)
        W2 = np.sin(-2*np.pi*m[j]/L[j]*l)
        A = 0.0
        for k in range(int(Kj)):
            G = data[int(k*Dj):int(k*Dj+L[j])].copy()
            G = G-np.mean(G)
            G = G*w
            A += np.dot(G,W1)**2 + np.dot(G,W2)**2
        pxx[j] = A/Kj*C_PSD

    return pxx,f

if __name__ =='__main__':
    fs = 10
    x = np.random.randn(1000)*np.sqrt(fs/2)
    pxx,f = lpsd(x,fs)

    plt.figure
    plt.loglog(f,np.sqrt(pxx))
    plt.grid()
    plt.xlabel('Frequency (Hz)')
    plt.ylabel(r'PSD ($\rm*/\sqrt{Hz}$)')
\end{lstlisting}


{\noindent \bfseries MATLAB Version}


\begin{lstlisting}[language=Matlab]
% Use LPSD mothod to plot power spectral density
% [Pxx,f] = iLPSD(Data,fs,Jdes)
%    Data --- Input data, processed by column
%    fs   --- Sample frequency, unit: Hz
%    Jdes --- Desired frequency points, default: 1000
%    Pxx  --- One-sided PSD, unit: *^2/Hz
%    f    --- Frequency points related to PSD points,unit: Hz
%    Default window function is hanning window.
% Demo:
%    iLPSD(data,fs)
%       Plot PSD using default settings.
%    h = iLPSD(data,fs,Jdes)
%       Plot PSD with desired points of 1000
%    [Pxx,f] = iLPSD(data,fs)
%       Return PSD points, not plot any figure

% Ref: Improved spectrum estimation from digitizedtime series on a logarithmic frequency axis
%      Article DOI: 10.1016/j.measurement.2005.10.010

% XiaoCY 2020-04-21

%% Main
function varargout = iLPSD(varargin)
    
    nargoutchk(0,2);
    narginchk(2,3);
    
    data = varargin{1};
    fs = varargin{2};
    if nargin == 3
        Jdes = varargin{3};
    else
        Jdes = 1000;
    end
    
    Kdes = 100;
    ksai = 0.5;
    
    [N,nCol] = size(data);
    if N==1 && nCol~=1
       data = data';
       N = nCol;
       nCol = 1;
    end
    
    [f,L,m] = getFreqs(N,fs,Jdes,Kdes,ksai);
    
    J = length(f);
    P = zeros(J,nCol);
    for j = 1:J
        Dj = floor((1-ksai)*L(j));
        Kj = floor((N-L(j))/Dj+1);
        w = hann(L(j));
        C_PSD = 2/fs/sum(w.^2);
        l = (0:L(j)-1)';
        W1 = cos(-2*pi*m(j)/L(j).*l);
        W2 = sin(-2*pi*m(j)/L(j).*l);
        A = zeros(1,nCol);
        for k = 0:Kj-1
            G = data(k*Dj+1:k*Dj+L(j),:);
            G = G-mean(G);        % G = detrend(G);   
            G = G.*w;
            A = A + sum(G.*W1).^2+sum(G.*W2).^2;
        end
        P(j,:) = A/Kj*C_PSD;
    end
    
    switch nargout
        case 0
            PlotPSD(P,f)
        case 1
            varargout{1} = PlotPSD(P,f);
        case 2
            varargout{1} = P;
            varargout{2} = f;
        otherwise
            % Do Nothing
    end
end

%% Subfunctions
% get logarithmic frequency points
function [f,L,m] = getFreqs(N,fs,Jdes,Kdes,ksai)
    fmin = fs/N;
    fmax = fs/2;
    r_avg = fs/N*(1+(1-ksai)*(Kdes-1));
    
    g = (N/2)^(1/(Jdes-1))-1;
    
    f = zeros(Jdes,1)-1;
    L = f;
    m = f;
    j = 1;
    fj = fmin;
    while fj < fmax
        rj = fj*g;
        if rj < r_avg
            rj = sqrt(rj*r_avg);
        end
        if rj < fmin
            rj = fmin;
        end
        
        Lj = floor(fs/rj);
        rj = fs/Lj;
        mj = fj/rj;
        
        f(j) = fj;
        L(j) = Lj;
        m(j) = mj;
        
        fj = fj+rj;
        j = j+1;
    end
    f(f<0) = [];
    L(L<0) = [];
    m(m<0) = [];
end

% plot PSD
function varargout = PlotPSD(P,f)
    hLine = loglog(f,sqrt(P));
    grid on
    xlabel('Frequency (Hz)')
    ylabel('PSD ([Unit]/Hz^{1/2})')
    
    if nargout == 1
        varargout{1} = hLine;
    end
end
\end{lstlisting}





\subsection{Main Code}


{\noindent \bfseries Python Version}


\begin{lstlisting}[language=Python]
'''
Main code for DSP project
(See ../DSP_Project_Reqirement&Guidance(2020 Fall).pdfProject-1 for detail.)

XiaoCY 2021-02-18
'''

#%%
import numpy as np
import matplotlib.pyplot as plt
import scipy.signal as signal
import springlib as sp

savepdf = False
N = 1000
fs = 1000.0
pi = np.pi

t = np.arange(N)/fs
x1 = np.cos(2*pi*50*t)
x2 = 0.3*np.cos(2*pi*100*t + pi/3)
x3 = 0.2*np.cos(2*pi*150*t - pi/3)
x4 = 0.1*np.sin(2*pi*250*t)
xn = 0.05*np.random.randn(N)
x = x1+x2+x3+x4+xn

#%% design IIR filter
Wp = 110./(fs/2)             # passband cornerfrequency (normalized)
Ws = 130./(fs/2)             # stopband cornerfrequency (normalized)
Rp = 1.                      # passband ripple (dB)
Rs = 60.                     # stopband attenuation(dB)

Nz,Wn = signal.cheb2ord(Wp,Ws,Rp,Rs)
b,a = signal.cheby2(Nz,Rs,Wn)

f = np.linspace(0,fs/2,500)
_,g = signal.freqz(b,a,f,fs=fs)
gdb = 20*np.log10(np.abs(g))

plt.figure()
plt.plot(f,gdb,label='Chebyshev Type II')
plt.grid()
plt.legend()
plt.xlabel('Frequency (Hz)')
plt.ylabel('Gain (dB)')
plt.xlim(0,fs/2)
if savepdf: 
    plt.savefig('prjFilter.pdf')
plt.show()

#%% filter data
y = signal.filtfilt(b,a,x)

plt.figure()
plt.plot(t,x,label='original')
plt.plot(t,y,label='filtered')
plt.grid()
plt.legend()
plt.xlabel('Time (s)')
plt.ylabel('Signal (A)')
plt.xlim(0,0.2)
if savepdf:
    plt.savefig('prjSignal.pdf')
plt.show()


pxx,fx = sp.lpsd(x,fs)
pyy,fy = sp.lpsd(y,fs)

plt.figure()
plt.semilogy(fx,np.sqrt(pxx),label='original')
plt.semilogy(fy,np.sqrt(pyy),label='filtered')
plt.grid()
plt.legend()
plt.xlabel('Frequency (Hz)')
plt.ylabel(r'Signal ($\rm A/\sqrt{Hz}$)')
plt.xlim(0,fs/2)
if savepdf:
    plt.savefig('prjPSD.pdf')
plt.show()
\end{lstlisting}


{\noindent \bfseries MATLAB Version}


\begin{lstlisting}[language=Matlab]
% Main code for DSP project
% (See ../DSP_Project_Reqirement&Guidance(2020 Fall).pdf Project-1 for detail.)

% XiaoCY 2021-02-18

%%
clear;clc

N = 1e3;
fs = 1e3;

t = (0:N-1)'/fs;
x1 = cos(2*pi*50*t);
x2 = 0.3*cos(2*pi*100*t + pi/3);
x3 = 0.2*cos(2*pi*150*t - pi/3);
x4 = 0.1*sin(2*pi*250*t);
xn = 0.05*randn(N,1);
x = x1+x2+x3+x4+xn;

%%
Wp = 110/(fs/2);                % passband corner frequency (normalized)
Ws = 130/(fs/2);                % stopband corner frequency (normalized)
Rp = 1;                         % passband ripple (dB)
Rs = 60;                        % stopband attenuation (dB)

[Nz,Wn] = cheb2ord(Wp,Ws,Rp,Rs);
[b,a] = cheby2(Nz,Rs,Wn);

f = linspace(0,fs/2,500);
g = freqz(b,a,f,fs);
gdb = 20*log10(abs(g));

figure('Name','filter')
plot(f,gdb,'DisplayName','Chebyshev Type II')
grid on
legend
xlabel('Frequency (Hz)')
ylabel('Gain (dB)')

%%
y = filtfilt(b,a,x);

figure('Name','time')
plot(t,x,'DisplayName','original')
hold on
grid on
plot(t,y,'DisplayName','filtered')
legend
xlabel('Time (s)')
ylabel('Signal (A)')
xlim([0,0.1])

figure('Name','PSD')
[pxx,f] = iLPSD([x,y],fs);
semilogy(f,sqrt(pxx))
grid on
legend('original','filtered')
xlabel('Frequency (Hz)')
ylabel('PSD (A/Hz^{1/2})')
\end{lstlisting}