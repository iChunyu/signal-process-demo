\section{Introduction}


Fourier series (FS) gives a new idea to expand a continuous and periodic signal with a series of trigonometric functions, which glanced at frequency analysis at the first time. Based on the idea of FS, other transformations were proposed to extend the limit on FS: (1) Fourier transform (FT) for continuous but non-periodic signals; (2) discrete time Fourier transform (DTFT) for sampled signals; (3) discrete Fourier transform (DFT) for completely discrete analysis. These transformations make up Fourier transform family under Dirichlet conditions. Besides, Laplace transform extends the frequency into complex domain and Z transform simplifies DTFT practically.


Analysis in frequency domain provides more options to process signals or evaluate linear systems. For example, a spectrum given by fast Fourier transform (FFT, the fast algorithm for DFT) shows the frequency distribution of the signal; the location of poles and zeros of a system demonstrate the stability and dynamic properties.


In this report, attentions are paid to digital signal processing. Power spectrum density (PSD) is firstly introduced as a tool for frequency analysis and a new algorithm is delivered. Then the design and properties of some common used filters are discussed. Finally, an example is given to show processing procedure and the results.


\section{Power Spectrum Density} \label{sec:psd}
\subsection{Classical Spectrum Estimation}
\subsection{LPSD Algorithm}

\section{Filter Design} \label{sec:filter}
\subsection{Common Used Filters}
% continus/discrete, FIR/IIR
% butter/... (functions)

\subsection{Zero-Phase Filtering}
% phase compensator
% flip/filter...

\section{Results}
% example

\section{Conclusions}

Text



